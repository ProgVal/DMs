\documentclass{article}

\usepackage[utf8]{inputenc}
\usepackage{amsmath}
\usepackage{amssymb}
\usepackage[left=2cm,right=2cm,top=2cm,bottom=2cm]{geometry}

\newcommand{\PP}[1]{\mathbb{P}\left[ #1 \right ]}
\newcommand{\EE}[1]{\mathbb{E}\left[ #1 \right ]}
\newcommand{\ZZ}{\mathbb{Z}}

\title{Optimisation -- Homework 2}
\author{Charrondière Raphaël}
\date{23 Avril 2015}
\begin{document}
\maketitle
\section*{Exercice 1 - MINIMAL-WEIGHT-SET-COVER}
\subsection*{Relaxation fractionnaire}
Variables : $\forall s\in S\ 0 \leq x_s$

Minimiser $$\sum_{s \in S} w_s \cdot x_s$$

Sous les conditions : $$\forall e \in E \sum_{s \in S | e \in s} x_s \geq 1$$

\subsection*{Dual}
Variables : $\forall e\in E\ 0 \leq y_e$

Maximiser $$\sum_{e \in E} y_e$$

Sous les conditions : $$\forall s \in S \sum_{e \in E | e \in s} y_e \leq w_s$$

\subsection*{Interprétation}

Si la fonction poids est 1.

Pour le primal, $x_s$ est 0 ou 1 suivant si le sous ensemble s est choisi pour la couverture, on minimise  le nombre de sous-ensembles choisis (fonction objectif), en imposant que chaque point soit dans au moins 1 des ensembles choisis.

Pour le dual, $y_e=1$ si $e$ est choisit ou $0$ sinon, on maximise le nombre de points choisis (fonction objectif) tel que deux points choisis ne soient pas dans le même sous ensemble de S, en bref on cherche à séparer ces ensembles.

\subsection*{Différence solution réelle-entière}

Objectif : trouver $f$ tel que $SC*\leq f(R-SC)$

\subsubsection*{Si les ensembles de S sont bornés}
 Notons $N_0=\max{|s|\ s\in S}$

$f=N_0$ répond au problème. En effet il on prend l'optimum réel, en multipliant la solution par $N_0$, la condition sur $e$ :  $\sum_{s \in S | e \in s} x_s \geq N_0$ impose qu'il existe un des termes de la somme plus grand que 1 (par définition de $N_0$). On peut donc prendre la partie entière des $x_s$ et respecter le système d'origine, l'objectif de cette solution est alors inférieur à $N_0$ fois l'objectif du système réel optimal.

\subsubsection*{Si les ensembles de S ont une taille $\frac{n}{2}$}

Choisissons $S_0$ formés de $\log n$ éléments de $S$ piochés uniformément.

$\forall x \in E \ \forall s \in S \ \PP{x \in s}=\frac{1}{2}$

$\forall x \in E \ \PP{x \in \bigcup_{s\in S_0} s}=\sum_{s_i in S_0}\PP{x \in s_i \text{ et } \forall t<i\ x \notin s_t }=\sum_{i=1}^{\log n}{\frac{1}{2^i}}=1-\frac{1}{2^{\log n}}=1-\frac{1}{n}=\frac{n-1}{n}$

On prend alors $\forall i\ x_i=\frac{1}{n-1}\rightarrow 1$, ce qui en moyenne valide les conditions pour l'instance réelle.

Ainsi $\EE{R-SC*}\leq \frac{n}{n-1}$

Pour l'instance entière :

$\PP{E=\bigcup_{s\in S_0} s}={\left(\frac{n-1}{n}\right)}^n \rightarrow \frac{1}{e}$

Ainsi avec probabilité non négligeable, $SC* \leq \log n  (R-SC*)$

\subsubsection*{Contre-exemple}

\section*{Exercice 2 - Bland's rule}
\subsection*{Question a}

$x_j=0$ si $j\in \bar{B}\setminus s$ et $x_s=y$

Pour $i \in B$ $x_i=b_i-\sum_{j\in\bar{B}}a_{ij}x_j= b_i-a_{is}y$

Dans D : $z=v+\sum_{j\in \bar{B}}c_j x_j=v+c_s y$

Dans D* : $z=v+\sum_{j\in \bar{B}}c_j^* x_j+\sum_{j\in B}c_j^* x_j=v+c_s^* y+\sum_{j\in B}c_j^* x_j=v+c_s^* y+\sum_{j\in B}c_j^* (b_j-a_{js}y)$

Ici, l'objectif est invariant donc :

$v+c_s y=v+c_s^* y+\sum_{j\in B}c_j^* (b_j-a_{js}y)$

\subsection*{Question b}

Montrons $\sum_{j\in B}c_j^* b_j=0$. En fait si ce n'est pas le cas l'objectif augmente, donc on n'est pas dans le cyle.

Ensuite $y\neq 0$ sinon $z = v$ et le simplexe a finit.

Donc de a on a $y\left (c_s-c_s*+\sum_{j\in B}c_j^* a_{js}\right )=0$

D'où $c_s-c_s*+\sum_{j\in B}c_j^* a_{js}=0$


\subsection*{Question c}

Si $c_s\leq 0$ alors le simplex a fini, ce qui n'est pas le cas donc $c_s>0$

\subsection*{Question d}

$s$ étant le plus petit index sortant $c_s^*<0$, sinon $x_s$ entre dans D*

\subsection*{Question e} 

Utilisant b, c et d on a immédiatement $\sum_{j\in B}c_j^* a_{js}<0$, donc il existe $r \in B$ tel que $c_r^* a_{rs}<0$

\subsection*{Question f} 

\subsection*{Question g} 

\subsection*{Question h} 

De même qu'en d, comme $r\neq t$ $c_r^* \leq 0$ en donc (e) $a_{rs}>0$.

\subsection*{Question i} 

\subsection*{Question j} 

%Comme les variables sont toutes positives, $\forall i b_i>0$. Donc xmq

%On a $y\geq 0$ 


\section*{Exercice 3}

Variables :
\begin{description}
\item[Gains] $g_1,g_2,g_3,g_4$
\item[Nombre fabricants] $b_1,b_2,b_3,b_4\geq 0$
\item[Nombre formeurs] $t_1,t_2,t_3\geq 0$
\item[Nombre formés] $s_1,s_2,s_3\geq 0$
\item[Nombre ouvriers] $w_1,w_2,w_3,w_4\geq 0$
\item[Constantes]
\begin{itemize}
\item
\item[$w_0$] nombre initial de salariés = 40
\item[$p_i$] gain net par téléphone fabriqué (payé par le client la ième semaine moins le prix de fabrication.)
\end{itemize}
\end{description}


Objectif : $$\text{Maximise }g_1+g_2+g_3+g_4$$

Conditions : 

Sur les employés :
$$w_i =  w_{i-1}+s_i$$
$$b_i+t_i+s_i\leq w_i$$
$$s_i=30\cdot t_i$$

Sur les gains :
$$g_i = 50\cdot p_i \cdot b_i-200\cdot w_i+100\cdot s_i$$
$$b_1+b_2+b_3+b_4\leq \frac{20000}{50}$$


\end{document}